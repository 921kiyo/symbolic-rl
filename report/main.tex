\documentclass[12pt,twoside]{report}
\usepackage{fixltx2e}
\usepackage{hyperref}
%%%%%%%%%%%%%%%%%%%%%%%%%%%%%%%%%%%%%%%%%%%%%%%%%%%%%%%%%%%%%%%%%%%%%%%%%%%%%

% Definitions for the title page
% Edit these to provide the correct information
% e.g. \newcommand{\reportauthor}{Timothy Kimber}

\newcommand{\reporttitle}{Data Efficient Deep Reinforcement Learning using Inductive Logic Programming}
\newcommand{\reportauthor}{Kiyohito Kunii}
\newcommand{\supervisor}{Professor Alessandra Russo}
\newcommand{\degreetype}{MSc in Computing Science}

%%%%%%%%%%%%%%%%%%%%%%%%%%%%%%%%%%%%%%%%%%%%%%%%%%%%%%%%%%%%%%%%%%%%%%%%%%%%%

% load some definitions and default packages
%%%%%%%%%%%%%%%%%%%%%%%%%%%%%%%%%%%%%%%%%
% University Assignment Title Page 
% LaTeX Template
% Version 1.0 (27/12/12)
%
% This template has been downloaded from:
% http://www.LaTeXTemplates.com
%
% Original author:
% WikiBooks (http://en.wikibooks.org/wiki/LaTeX/Title_Creation)
%
% License:
% CC BY-NC-SA 3.0 (http://creativecommons.org/licenses/by-nc-sa/3.0/)
% 
%
%%%%%%%%%%%%%%%%%%%%%%%%%%%%%%%%%%%%%%%%%
%----------------------------------------------------------------------------------------
%	PACKAGES AND OTHER DOCUMENT CONFIGURATIONS
%----------------------------------------------------------------------------------------
\usepackage[a4paper,hmargin=2.8cm,vmargin=2.0cm,includeheadfoot]{geometry}
\usepackage{textpos}
\usepackage{natbib} % for bibliography
\usepackage{tabularx,longtable,multirow,subfigure,caption}%hangcaption
\usepackage{fncylab} %formatting of labels
\usepackage{fancyhdr} % page layout
\usepackage{url} % URLs
\usepackage[english]{babel}
\usepackage{amsmath}
\usepackage{graphicx}
\usepackage{dsfont}
\usepackage{epstopdf} % automatically replace .eps with .pdf in graphics
\usepackage{backref} % needed for citations
\usepackage{array}
\usepackage{latexsym}
\usepackage[pdftex,pagebackref,hypertexnames=false,colorlinks]{hyperref} % provide links in pdf

\hypersetup{pdftitle={},
  pdfsubject={}, 
  pdfauthor={},
  pdfkeywords={}, 
  pdfstartview=FitH,
  pdfpagemode={UseOutlines},% None, FullScreen, UseOutlines
  bookmarksnumbered=true, bookmarksopen=true, colorlinks,
    citecolor=black,%
    filecolor=black,%
    linkcolor=black,%
    urlcolor=black}

\usepackage[all]{hypcap}


%\usepackage{color}
%\usepackage[tight,ugly]{units}
%\usepackage{float}
%\usepackage{tcolorbox}
%\usepackage[colorinlistoftodos]{todonotes}
% \usepackage{ntheorem}
% \theoremstyle{break}
% \newtheorem{lemma}{Lemma}
% \newtheorem{theorem}{Theorem}
% \newtheorem{remark}{Remark}
% \newtheorem{definition}{Definition}
% \newtheorem{proof}{Proof}


%%% Default fonts
\renewcommand*{\rmdefault}{bch}
\renewcommand*{\ttdefault}{cmtt}



%%% Default settings (page layout)
\setlength{\parindent}{0em}  % indentation of paragraph

\setlength{\headheight}{14.5pt}
\pagestyle{fancy}
\renewcommand{\chaptermark}[1]{\markboth{\chaptername\ \thechapter.\ #1}{}} 

\fancyfoot[ER,OL]{\sffamily\textbf{\thepage}}%Page no. in the left on odd pages and on right on even pages
\fancyfoot[OC,EC]{\sffamily }
\renewcommand{\headrulewidth}{0.1pt}
\renewcommand{\footrulewidth}{0.1pt}
\captionsetup{margin=10pt,font=small,labelfont=bf}


%--- chapter heading

\def\@makechapterhead#1{%
  \vspace*{10\p@}%
  {\parindent \z@ \raggedright \sffamily
    \interlinepenalty\@M
    \Huge\bfseries \thechapter \space\space #1\par\nobreak
    \vskip 30\p@
  }}

%---chapter heading for \chapter*  
\def\@makeschapterhead#1{%
  \vspace*{10\p@}%
  {\parindent \z@ \raggedright
    \sffamily
    \interlinepenalty\@M
    \Huge \bfseries  #1\par\nobreak
    \vskip 30\p@
  }}

\allowdisplaybreaks

% load some macros
% Here, you can define your own macros. Some examples are given below.

\newcommand{\R}[0]{\mathds{R}} % real numbers
\newcommand{\Z}[0]{\mathds{Z}} % integers
\newcommand{\N}[0]{\mathds{N}} % natural numbers
\newcommand{\C}[0]{\mathds{C}} % complex numbers
\renewcommand{\vec}[1]{{\boldsymbol{{#1}}}} % vector
\newcommand{\mat}[1]{{\boldsymbol{{#1}}}} % matrix


\date{June 2018}

\begin{document}

% load title page
% Last modification: 2015-08-17 (Marc Deisenroth)
\begin{titlepage}

\newcommand{\HRule}{\rule{\linewidth}{0.5mm}} % Defines a new command for the horizontal lines, change thickness here


%----------------------------------------------------------------------------------------
%	LOGO SECTION
%----------------------------------------------------------------------------------------

\includegraphics[width = 4cm]{./figures/imperial}\\[0.5cm] 

\center % Center remainder of the page

%----------------------------------------------------------------------------------------
%	HEADING SECTIONS
%----------------------------------------------------------------------------------------

\textsc{\Large Imperial College London}\\[0.5cm] 
\textsc{\large Department of Computing}\\[0.5cm] 

%----------------------------------------------------------------------------------------
%	TITLE SECTION
%----------------------------------------------------------------------------------------

\HRule \\[0.4cm]
{ \huge \bfseries \reporttitle}\\ % Title of your document
\HRule \\[1.5cm]
 
%----------------------------------------------------------------------------------------
%	AUTHOR SECTION
%----------------------------------------------------------------------------------------

\begin{minipage}{0.4\textwidth}
\begin{flushleft} \large
\emph{Author:}\\
\reportauthor % Your name
\end{flushleft}
\end{minipage}
~
\begin{minipage}{0.4\textwidth}
\begin{flushright} \large
\emph{Supervisor:} \\
\supervisor % Supervisor's Name
\end{flushright}
\end{minipage}\\[4cm]


%----------------------------------------------------------------------------------------
%	FOOTER & DATE SECTION
%----------------------------------------------------------------------------------------
\vfill % Fill the rest of the page with whitespace
Submitted in partial fulfillment of the requirements for the MSc degree in
\degreetype~of Imperial College London\\[0.5cm]

\makeatletter
\@date 
\makeatother


\end{titlepage}



% page numbering etc.
\pagenumbering{roman}
\clearpage{\pagestyle{empty}\cleardoublepage}
\setcounter{page}{1}
\pagestyle{fancy}

%%%%%%%%%%%%%%%%%%%%%%%%%%%%%%%%%%%%
% \begin{abstract}
% Your abstract.
% \end{abstract}
%
% \cleardoublepage
%%%%%%%%%%%%%%%%%%%%%%%%%%%%%%%%%%%%
% \section*{Acknowledgments}
% Comment this out if not needed.
%
% \clearpage{\pagestyle{empty}\cleardoublepag e}

%%%%%%%%%%%%%%%%%%%%%%%%%%%%%%%%%%%%
%--- table of contents
\fancyhead[RE,LO]{\sffamily {Table of Contents}}
\tableofcontents


\clearpage{\pagestyle{empty}\cleardoublepage}
\pagenumbering{arabic}
\setcounter{page}{1}
\fancyhead[LE,RO]{\slshape \rightmark}
\fancyhead[LO,RE]{\slshape \leftmark}

%%%%%%%%%%%%%%%%%%%%%%%%%%%%%%%%%%%%
\chapter{Introduction}

% IMPERIAL LOGO
% \begin{figure}[tb]
% \centering
% \includegraphics[width = 0.4\hsize]{./figures/imperial}
% \caption{Imperial College Logo. It's nice blue, and the font is quite stylish. But you can choose a different one if you don't like it.}
% \label{fig:logo}
% \end{figure}
% Figure~\ref{fig:logo} is an example of a figure.


There has been successful applications of deep reinforcement learning (DRL) a number of domains, such as games and robotics.
DRL is considered to be a step towards artificial general intelligence.
However there are still a number of issues to overcome in this method. As pointed by XXX, there are 3 major problems.
First, it requires a large amount of data for training the model, which requires a long time of learning process.
Second, it is considered to be a black-box, meaning the decision making process is unknown to human and therefore not explanable.
Third there is no thoughts process to the decision making, which, as XX points out, is a fundamental to the artificial general intelligence.
To overcome these problems, there are 3 main streams of research on this field.
First main research is focused on applying Baysian statistics XXX, the second main research is XXX.
Recently, the XXX attempted to incorporate symbolic representations into the system to achive more data-efficient learning, which shows a promissing results of this approach.
The paper was the application of symbolic representatiojns into a very simple game to demonstrate this proof of concept would actually work.


By contract, there is active reserach in symbolic machine learning, which focuses on logic=based learning rather than statistical machine learning.
For example, XX shows the agent can learn XXX from a noisy examples, with only a very few training examples.


In this paper, I explore incorporation of symbolic machine learning into XXX to achieve data-efficient learning using Inductive Learning of Answer Set Programs (ILASP), which is the state-of-art symbolic learning method that can be applied to imcomplete and more complex environment.


This research is based on XX, but in this paper I explore symbolic represetations process and include more learning aspect.


Problem setting \cite{garnelo2016towards}

%%%%%%%%%%%%%%%%%%%%%%%%%%%%%%%%%%%
\chapter{Background}

\section{Data Efficient Learning}

Two most studied approach for using previous learning exprience is meta-learning and transfer learning

Artificial General Intelligence

Definision of AGI

The history of data-efficient learning
What other people have done in this.

The advance of statistical machine learning methods, especially deeep reinforcement learning

AlphaGo, and AlphaGo Zero

Also business success.

Study of symbolic machine learning roots from
% Relational Reinforcement Learning

Baysian Optimisation

RNN approach

Symbolic Deep reinforcement learning

Some implementation: German paper

\section{ASP in Reinforcement Learning}

\chapter{Background}
\section{Inductive Logic Programming}
Inductive Logic Programming (ILP) is a subfield of research area aimed at the intersection of machine learning and logic programming (MUGGLETON 1991). The purpose of ILP is to derive a hypothesis H that is a solution of a learning task, which coveres all positive examples and any of negative examples.

Herbrand Model

Least Herbrand Model

Definite Logic Program is a set of definite rule where

\begin{theorem}
a definite rule is of the form h \leftarrow a\textsubscript{1}, ..., a\textsubscript{n}, where h , a\textsubscript{1}, ..., a\textsubscript{n} are all atoms.
\end{theorem}

whereas

Normal Logic Program is a set of normal rule where

\begin{theorem}
a normal rule is of the form h \leftarrow a\textsubscript{1}, ..., a\textsubscript{n}, not b\textsubscript{1}, ..., not b\textsubscript{n} where h is the head of the rule,
 and a\textsubscript{1}, ..., a\textsubscript{n}, b\textsubscript{1}, ..., b\textsubscript{n} are the body of the rule (both the head and body are all atoms).
\end{theorem}

\subsection{Stable Model Semantics}

Stable Model of a normal logic program

\subsection{Anwer Set Programming}

Literals

Negation as a failure

constrains is of the form
\begin{theorem}
\leftarrow a\textsubscript{1}, ..., a\textsubscript{n}, not b\textsubscript{1}, ..., not b\textsubscript{n}
\end{theorem}

Constraints are to filtering any irrelevant answer sets.

Syntax Examples

There are two types of constraints: soft and hard constraints. Soft constraint is XXX

Hard constraint is XXX

optimisation statement is of the form.

Which is useful to order the answer sets in terms of preference.

Syntax examples.

Aggregate

choice rules

An ASP program P is normal logic program with addition of choice rules, constraints and optimisation statement.

Answer set of P is


\subsubsection{Cautious Induction}
Cautious inductiion is based on the cautious semantics of answer set programs.

Sakama 2008

no concept of negative examples.


\subsubsection{Brave Induction}

Brave Induction is


Brave indction cannot learn constrains


Sakama 2009

\subsection{Learning from Anwer Sets (LAS)}

Learning from Answer Sets was developed in XX to overcome limitations of cautious induction and brave induction.


Partial Interpretations of the form



\subsection{Inductive Learning of Answer Set Programs (ILASP)}

ILASP is an algorithm that is capable of solving LAS tasks

It is based on two fundamental concepts: positive solutions and violating solutions.
A hypothesis H is a positive solution if and only if
1. H \subseteq S\textsubscript{M} \\
2. \forall e\textsuperscript{+} \in \exists A \in AS(B \cup H) subject to A extends e\textsuperscript{+}\\

A hypothesis H is a violating solution if and only if
1. H \subseteq S\textsubscript{M} \\
2. \forall e\textsuperscript{+} \in E\textsuperscript{+} \exists A \in AS(B \cup H) subject to A extends e\textsuperscript{+}\\
3. \exists e\textsuperscript{-} \in E\textsuperscript{-} \exists A \in AS(B \cup H) subject to A extends e\textsuperscript{-}\\


ILP\textsubscript{LAS} is positive solutions that are not violating solutions. 

ILASP task containing a contex-dependent example

TODO Explain how symbolic learning works

TODO What would you learn in my context? Relashinship of the objects?
Objects, types, locations and interactions.

\section{Reinforcement Learning}



\section{Symbolic Reinforcement Learning}


Explain the paper
\section{GVGAI Framework}


\chapter{Project Overview}
\section{Motivation}

Why symbolic reinforcement learning is good attempt

Reason 1: Complehensive by humans -> Explanable rather than black-box
Reason 2: Similar to the human, it uses reasoning

Use of previous experience (background knowledge)

Not much explored yet.

However there is a room for exploration on this field.

TODO Explain how reinforcement learning works

Reason 3: Recent advance of ILASP is promissing

Because of the recent advancement of logic-based learning and deep reinforcement learning, combination of both approach would be a next explonation toward artificial general intelligence.


\section{Objectives}

combining the two novel approaches to overcome the problems of the QND

\section{Project outline}

The probject outline

Implement baseline performance (DQN)


Pipeline
Implement the CNN side that is able to extract features of the game and convert into ASP syntax.


Apply ILASP to the ASP, which involves development of the pipeline of ILASP in Python

Finally use Q-learning that allow



which measurement would you use? (grid word, something else? GVGAL games)
Summarise different types of knowledge representations (Objects ?? relationship?)

Common sense

Implement based on \"Towards Deep Symbolic Reinforcement Learning \"

\section{Contribution}


\section{Legal and Ethical Issues}

???


%%%%%%%%%%%%%%%%%%%%%%%%%%%%%%%%%%%%
% \chapter{Contribution}


% %%%%%%%%%%%%%%%%%%%%%%%%%%%%%%%%%%%%
% \chapter{Experimental Results}
%
%
% %%%%%%%%%%%%%%%%%%%%%%%%%%%%%%%%%%%%
% \chapter{Conclusion}

\section{References}
%% bibliography
\bibliographystyle{apa}
\bibliography{references}

\end{document}
