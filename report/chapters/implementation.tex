Once the architecture is decided, we will implement it using Python and ILASPv3.1.0 (Clingo) \cite{Law2017}, and compare the performance with other learning methods using a game platform called GVG-AI Framework.

Other learning methods to be compared as benchmarks will be Q-Learning and Deep Q-Learning since these methods are widely used in other related works. Other symbolic-based approaches could also be compared as an extension. The two main measurements for the performance of our new architecture are learning efficiency and transfer learning capability as stated in the Introduction.

GVG-AI Framework was created for the General Video Gamea AI Competition \footnote{http://www.gvgai.net/}, a game environment for an agent that should be able to play a wide variety of games without knowing which games are to be played.
The underlying language is the Video Game Definition Language (VGDL), which is a high-level description language for 2D video games providing a platform for computational intelligence research (\cite{Schaul2013}).

TODO OPENAI paper citation

The game is formalised as MDP as follows:

\begin{itemize}

\item States: XXX
\item Actions: The agent can move up, down, right or left
\item Rewards: XXX
\item Transitions: XXX

\end{itemize}
